
\chapter{Papers}

\section{Hackman and Katz 2010}

\textbf{Group behavior and performance:} Dates the psychology of small
groups to the 1940s, motivated by military concerns ``for example, to
identify the factors that shaped the performance of infantry squads in
combat, or to properly design and lead groups that provided back-home
support for the war effort.''  Current: ``Although group behaviour and
performance once again is a highly active field of study, it has moved
out of its ancestral home in social psychology.  As social
psychologists increasingly have drawn upon cognitive neuroscience and
evolutionary theories to explain social phenomena, small group
research has migrated to the periphery of the field.''  Talks about
the fact that what a group is is changing, in response to changed
technologies, which enable groups to form in new ways, and with new
types of bonds.

``This chapter deals only with the behavior and performance of
\emph{purposive groups} --- that is, real groups [are there other
kinds?] that exist to accomplish something.''  Interesting to contrast
with groups which accidentally achieve things.  I wonder if Anonymous
fits into this description of a group?  

\textbf{Definition:} ``A group is an intact social system, complete
with boundaries, interdependence for some shared purpose, and
differentiated member roles.''  Hard to see much open source in this
--- the boundaries are very loose.  Not sure that Anonymous always has
differentiated member roles. 

``As social systems, groups are preceived as entities by both members
and nonmembers [not if they're secret], they create and redefine
realities...'' True: people start to anthropomorphize the group, and
are also interested in heuristics for understanding the group, and so
on. 

``[I]t is conceptually and empirically challenging to develop measures
of group performance that are both meaningful and psyscometrically
adequate.''

``Group experiences clearly can contribute positively to member
well-being, but they also can have the opposite effect when the group
is structured and led in ways that give collective accomplisment
priority over individual well-being.''

``Leavitt suggested that groups generate so many benefits that serious
consideration should be given to using groups rather than individuals
as the basic building blocks of organizations.''

\chapter{Further reading}

Groysberg, Healy, and Gui (2008): Very interesting result, suggesting
that for their task a small group (4-6 members) did something better
than both smaller and larger groups.

Janis, of course.  And Ostrom.

Schell.

Pentland.

``Suppose we took groups seriously'' --- Leavitt.  ``The importance of
the individual in an age of groupism'' (Locke et al, 2001).

Burt.

Ostrom.

West and Bettencourt

Scott Page

Hare 1976.

McGrath and Altman, 1966.

Ericsson.

Levine and Moreland (1998 review ``Small Groups'')

\chapter{Questions and ideas}

\emph{Q:} What's the analogue of the 10,000 hour rule for collective
intelligence?  If we view the 10,000 hour rule as primarily economic
--- a response to scarcity --- then things start to look very
interesting.  First, it is obvious that the 10,000 hour rule doesn't
always apply at the individual level (since different rules will
govern the scarcity).  And economics can be used to think about this
in a group setting, too.
