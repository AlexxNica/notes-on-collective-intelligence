
\chapter{Papers}

\section{Hackman and Katz 2010}

\textbf{Group behavior and performance:} Dates the psychology of small
groups to the 1940s, motivated by military concerns ``for example, to
identify the factors that shaped the performance of infantry squads in
combat, or to properly design and lead groups that provided back-home
support for the war effort.''  Current: ``Although group behaviour and
performance once again is a highly active field of study, it has moved
out of its ancestral home in social psychology.  As social
psychologists increasingly have drawn upon cognitive neuroscience and
evolutionary theories to explain social phenomena, small group
research has migrated to the periphery of the field.''  Talks about
the fact that what a group is is changing, in response to changed
technologies, which enable groups to form in new ways, and with new
types of bonds.





\chapter{Further reading}

Pentland.

Burt.

West and Bettencourt

Scott Page

Hare 1976.

McGrath and Altman, 1966.

Ericsson.

Levine and Moreland (1998 review ``Small Groups'')